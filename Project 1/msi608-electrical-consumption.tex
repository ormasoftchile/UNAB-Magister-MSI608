\documentclass{ieeeaccess}
\usepackage{cite}
\usepackage{amsmath,amssymb,amsfonts}
\usepackage{caption}
\usepackage{algorithmic}
\usepackage{graphicx}
\usepackage{textcomp}
% \def\BibTeX{{\rm B\kern-.05em{\sc i\kern-.025em b}\kern-.08em
%     T\kern-.1667em\lower.7ex\hbox{E}\kern-.125emX}}
% define graphics path
\graphicspath{ {./figures/} }
\begin{document}
% \history{Date of publication xxxx 00, 0000, date of current version xxxx 00, 0000.}
% \doi{10.1109/ACCESS.2017.DOI}

\title{(Draft) Electrical consumption prediction in central Chile: A machine learning approach}
\author{\uppercase{Cristián Ormazábal}\authorrefmark{1}, \IEEEmembership{Master candidate, UNAB}, (e-mail: cristian@ormasoft.cl).}

\begin{abstract}
    In central Chile, electricity demand has significantly increased due to factors
    like population growth, industrial expansion, and climate change, which affects consumption
    patterns. This project aims to predict the electrical consumption of several key substations,
    starting with Alto Jahuel, to better understand demand and improve energy planning.
    
\begin{quotation}
"Electric consumption is a major challenge for the management of energy infrastructure in 
    Chile. This study addresses the prediction of electric consumption in substations in the central
    region, starting with the Alto Jahuel substation. Accurate prediction will allow electric companies
    to optimize energy distribution, improve efficiency, and reduce the risk of overloads and blackouts."
    \end{quotation}
    
\end{abstract}

% \begin{keywords}
% Enter key words or phrases in alphabetical 
% order, separated by commas. For a list of suggested keywords, send a blank 
% e-mail to keywords@ieee.org or visit \underline
% {http://www.ieee.org/organizations/pubs/ani\_prod/keywrd98.txt}
% \end{keywords}

\titlepgskip=-15pt

\maketitle

\section{Introduction}
\label{sec:introduction}
\paragraph{Background} \PARstart{T}{he} electrical consumption of several key substations in Central 
Chile is a critical factor in energy planning. Electricity is a non-storable product, so accurate
 forecasting helps avoid issues with excessive production or those caused by lacking production.
  Power Load Forecasting (PLF) is classified in terms of the planning horizon’s duration: up to 1
   day/week ahead for short-term, 1 day/week to 1 year ahead for medium-term, and more than 1 year
    ahead for long-term \cite{b1}.

\paragraph{Research question} Devise a model to forecast the electrical consumption of several 
key substations in Central Chile, to better understand demand and improve energy planning. 
Electricity is non storable product, so accurate forecasting helps avoid issues with excessive 
production or those caused by lacking production. Power Load Forecasting is classified in terms
 of horizon: short term, mid term and long term. \footnote{The EPLF is classified in terms of
the planning horizon’s duration: up to 1 day/week ahead for short-term, 1 day/week to 1 year
ahead for medium-term, and more than 1 year ahead for long-term} \footnote{Excepteur sint,
sunt in culpa qui officia deserunt mollit anim id est laborum.}

\paragraph{Answer to the question} Demand for electricity is influenced by a variety of factors, including population growth, industrial expansion, and climate change. The project aims to predict the electrical consumption of several key substations, starting with Alto Jahuel, to better understand demand and improve energy planning. Accurate prediction will allow electric companies to optimize energy distribution, improve efficiency, and reduce the risk of overloads and blackouts.

\paragraph{ARIMA} The ARIMA models and their variants are the most widely used models for time series forecasting. The ARIMA model is a generalization of the ARMA model, which is a generalization of the AR model and the MA model. The AR model is a linear regression model that uses the dependent relationship between an observation and some number of lagged observations. The MA model is a linear regression model that uses the dependency between an observation and a residual error from a moving average model applied to lagged observations. The ARMA model combines both the AR and MA models. The ARIMA model adds differencing of the observations to make the time series stationary, which is a key requirement of the model.

\section{Study objective}\label{s:section}

The objective of this project is to accurately predict electricity consumption at the central Chile substations using machine learning techniques to improve the responsiveness of the electric system to demand fluctuations.

\begin{figure}[h]
\centering
\includegraphics[width=0.45\textwidth]{consumption_by_substation}
\caption{Train data for consumption by substation}
\label{fig:figure_consumption_by_substation}
\end{figure}

\section{Methodology}\label{s:methodology}

The methodology is based on the canonical machine learning pipeline. Regarding the model selection, a glimpse over \cite{MS15} and \cite{MS19} is recommended.
The methodology for the current project consists of the following steps:

\begin{enumerate}
\item Data collection: The data was collected from the central Chile substations.	
\item Data preprocessing: The data was preprocessed to remove any missing values and outliers.
\item Feature engineering: The data was transformed into features that could be used by the machine learning model.
\item Model selection: The model was selected based on its performance on the training data.
\item Model training: The model was trained on the training data.
\item Model evaluation: The model was evaluated on the test data.
% \item Model deployment: The model was deployed to predict the electrical consumption of the central Chile substations.
% \item Model monitoring: The model was monitored to ensure that it continued to perform well over time.
% \item Model updating: The model was updated as needed to improve its performance.
% \item Model maintenance: The model was maintained to ensure that it continued to perform well.
\end{enumerate}

\section{Experiments}\label{s:experiments}
\begin{table}
\caption{Substations statistics}
\begin{tabular}{lrrrr}
\toprule
substation & min & max & mean & std \\
\midrule
AJAHUEL & -159.02 & 348.89 & 157.71 & 53.69 \\
BUIN & -3.19 & 0.00 & -1.36 & 0.24 \\
CHENA & -74.50 & 486.40 & 191.42 & 48.96 \\
CNAVIA & -92.67 & 752.01 & 358.86 & 109.50 \\
ELSALTO & 0.00 & 749.89 & 370.65 & 124.57 \\
FLORIDA & -77.78 & 91.92 & -1.83 & 11.65 \\
LOSALME & -1.03 & 353.61 & 181.40 & 50.38 \\
\bottomrule
\end{tabular}
\end{table}

\begin{figure}[h]
    \centering
    \includegraphics[width=0.45\textwidth]{consumption_distribution}
    \caption{Consumption distribution}
    \label{fig:figure_consumption_distribution}
    \end{figure}

\begin{enumerate}
    \item Data analysis
    \begin {enumerate}
        \item {Data quality}: There are no missing values in the data.
        \item Data distribution: See \ref{fig:figure_consumption_distribution}. 
        \item Data visualization: See FIGURE 1 for visualizing the consumption by substation. For each plot we can appreciate there seems to be a cyclic pattern in the data that repeats every year.
    \end{enumerate}
    \item Descriptive statistics: See Table \ref{tab:table_substations}. For the BUIN substation, the standard
    deviation is very low. The extreme values -199 and 6.633003 appear as outliers, so they are going to be removed.
    \item Relevant graphs

The experiments were conducted using the Alto Jahuel substation data. 
The data was split into training and test sets, with 80\% of the data used for training and 20\%
 used for testing. The model was trained on the training data and evaluated on the test data.
 The results are shown in Figure \ref{fig:figure_consumption_by_substation}.

\begin{figure}[h]
	\centering
	\includegraphics[width=0.5\textwidth]{zscore_AJAHUEL_50}
	\caption{Z-score of the consumption of the Alto Jahuel substation}
	\label{fig:figure_zscore_AJAHUEL_50}
\end{figure}

\begin{figure}[h]
	\centering
	\includegraphics[width=0.5\textwidth]{decomposition_AJAHUEL_50_8760}
	\caption{Decomposition of the consumption of the Alto Jahuel substation}
	\label{fig:figure_decomposition_AJAHUEL_50_8760}
\end{figure}

\section{Publication Principles}
The two types of contents of that are published are; 1) peer-reviewed and 2) 
archival. The Access Department publishes scholarly 
articles of archival value as well as tutorial expositions and critical 
reviews of classical subjects and topics of current interest. 

Authors should consider the following points:

\begin{enumerate}
\item Technical papers submitted for publication must advance the state of knowledge and must cite relevant prior work. 
\item The length of a submitted paper should be commensurate with the importance, or appropriate to the complexity, of the work. For example, an obvious extension of previously published work might not be appropriate for publication or might be adequately treated in just a few pages.
\item Authors must convince both peer reviewers and the editors of the scientific and technical merit of a paper; the standards of proof are higher when extraordinary or unexpected results are reported. 
\item Because replication is required for scientific progress, papers submitted for publication must provide sufficient information to allow readers to perform similar experiments or calculations and 
use the reported results. Although not everything need be disclosed, a paper 
must contain new, useable, and fully described information. For example, a 
specimen's chemical composition need not be reported if the main purpose of 
a paper is to introduce a new measurement technique. Authors should expect 
to be challenged by reviewers if the results are not supported by adequate 
data and critical details.
\item Papers that describe ongoing work or announce the latest technical achievement, which are suitable for presentation at a professional conference, may not be appropriate for publication.
\end{enumerate}

\section{Reference Examples}

\begin{itemize}

\item \emph{Basic format for books:}\\
J. K. Author, ``Title of chapter in the book,'' in \emph{Title of His Published Book, x}th ed. City of Publisher, (only U.S. State), Country: Abbrev. of Publisher, year, ch. $x$, sec. $x$, pp. \emph{xxx--xxx.}\\
See \cite{b1,b2}.

\item \emph{Basic format for periodicals:}\\
J. K. Author, ``Name of paper,'' \emph{Abbrev. Title of Periodical}, vol. \emph{x, no}. $x, $pp\emph{. xxx--xxx, }Abbrev. Month, year, DOI. 10.1109.\emph{XXX}.123456.\\
See \cite{b3}--\cite{b5}.

\item \emph{Basic format for reports:}\\
J. K. Author, ``Title of report,'' Abbrev. Name of Co., City of Co., Abbrev. State, Country, Rep. \emph{xxx}, year.\\
See \cite{b6,b7}.

\item \emph{Basic format for handbooks:}\\
\emph{Name of Manual/Handbook, x} ed., Abbrev. Name of Co., City of Co., Abbrev. State, Country, year, pp. \emph{xxx--xxx.}\\
See \cite{b8,b9}.

\item \emph{Basic format for books (when available online):}\\
J. K. Author, ``Title of chapter in the book,'' in \emph{Title of
Published Book}, $x$th ed. City of Publisher, State, Country: Abbrev.
of Publisher, year, ch. $x$, sec. $x$, pp. \emph{xxx--xxx}. [Online].
Available: \underline{http://www.web.com}\\
See \cite{b10}--\cite{b13}.

\item \emph{Basic format for journals (when available online):}\\
J. K. Author, ``Name of paper,'' \emph{Abbrev. Title of Periodical}, vol. $x$, no. $x$, pp. \emph{xxx--xxx}, Abbrev. Month, year. Accessed on: Month, Day, year, DOI: 10.1109.\emph{XXX}.123456, [Online].\\
See \cite{b14}--\cite{b16}.

\item \emph{Basic format for papers presented at conferences (when available online): }\\
J.K. Author. (year, month). Title. presented at abbrev. conference title. [Type of Medium]. Available: site/path/file\\
See \cite{b17}.

\item \emph{Basic format for reports and handbooks (when available online):}\\
J. K. Author. ``Title of report,'' Company. City, State, Country. Rep. no., (optional: vol./issue), Date. [Online] Available: site/path/file\\
See \cite{b18,b19}.

\item \emph{Basic format for computer programs and electronic documents (when available online): }\\
Legislative body. Number of Congress, Session. (year, month day). \emph{Number of bill or resolution}, \emph{Title}. [Type of medium]. Available: site/path/file\\
\textbf{\emph{NOTE: }ISO recommends that capitalization follow the accepted practice for the language or script in which the information is given.}\\
See \cite{b20}.

\item \emph{Basic format for patents (when available online):}\\
Name of the invention, by inventor's name. (year, month day). Patent Number [Type of medium]. Available: site/path/file\\
See \cite{b21}.

\item \emph{Basic format}\emph{for conference proceedings (published):}\\
J. K. Author, ``Title of paper,'' in \emph{Abbreviated Name of Conf.}, City of Conf., Abbrev. State (if given), Country, year, pp. \emph{xxxxxx.}\\
See \cite{b22}.

\item \emph{Example for papers presented at conferences (unpublished):}\\
See \cite{b23}.

\item \emph{Basic format for patents}$:$\\
J. K. Author, ``Title of patent,'' U.S. Patent \emph{x xxx xxx}, Abbrev. Month, day, year.\\
See \cite{b24}.

\item \emph{Basic format for theses (M.S.) and dissertations (Ph.D.):}
\begin{enumerate}
\item J. K. Author, ``Title of thesis,'' M.S. thesis, Abbrev. Dept., Abbrev. Univ., City of Univ., Abbrev. State, year.
\item J. K. Author, ``Title of dissertation,'' Ph.D. dissertation, Abbrev. Dept., Abbrev. Univ., City of Univ., Abbrev. State, year.
\end{enumerate}
See \cite{b25,b26}.

\item \emph{Basic format for the most common types of unpublished references:}
\begin{enumerate}
\item J. K. Author, private communication, Abbrev. Month, year.
\item J. K. Author, ``Title of paper,'' unpublished.
\item J. K. Author, ``Title of paper,'' to be published.
\end{enumerate}
See \cite{b27}--\cite{b29}.

\item \emph{Basic formats for standards:}
\begin{enumerate}
\item \emph{Title of Standard}, Standard number, date.
\item \emph{Title of Standard}, Standard number, Corporate author, location, date.
\end{enumerate}
See \cite{b30,b31}.

\item \emph{Article number in~reference examples:}\\
See \cite{b32,b33}.

\item \emph{Example when using et al.:}\\
See \cite{b34}.

\end{itemize}

\begin{thebibliography}{00}

\bibitem{b1} G. O. Young, ``Synthetic structure of industrial plastics,'' in \emph{Plastics,} 2\textsuperscript{nd} ed., vol. 3, J. Peters, Ed. New York, NY, USA: McGraw-Hill, 1964, pp. 15--64.

\bibitem{b2} W.-K. Chen, \emph{Linear Networks and Systems.} Belmont, CA, USA: Wadsworth, 1993, pp. 123--135.

\bibitem{b3} J. U. Duncombe, ``Infrared navigation---Part I: An assessment of feasibility,'' \emph{IEEE Trans. Electron Devices}, vol. ED-11, no. 1, pp. 34--39, Jan. 1959, 10.1109/TED.2016.2628402.

\bibitem{b4} E. P. Wigner, ``Theory of traveling-wave optical laser,'' \emph{Phys. Rev}., vol. 134, pp. A635--A646, Dec. 1965.

\bibitem{b5} E. H. Miller, ``A note on reflector arrays,'' \emph{IEEE Trans. Antennas Propagat}., to be published.

\bibitem{b6} E. E. Reber, R. L. Michell, and C. J. Carter, ``Oxygen absorption in the earth's atmosphere,'' Aerospace Corp., Los Angeles, CA, USA, Tech. Rep. TR-0200 (4230-46)-3, Nov. 1988.

\bibitem{b7} J. H. Davis and J. R. Cogdell, ``Calibration program for the 16-foot antenna,'' Elect. Eng. Res. Lab., Univ. Texas, Austin, TX, USA, Tech. Memo. NGL-006-69-3, Nov. 15, 1987.

\bibitem{b8} \emph{Transmission Systems for Communications}, 3\textsuperscript{rd} ed., Western Electric Co., Winston-Salem, NC, USA, 1985, pp. 44--60.

\bibitem{b9} \emph{Motorola Semiconductor Data Manual}, Motorola Semiconductor Products Inc., Phoenix, AZ, USA, 1989.

\bibitem{b10} G. O. Young, ``Synthetic structure of industrial
plastics,'' in Plastics, vol. 3, Polymers of Hexadromicon, J. Peters,
Ed., 2\textsuperscript{nd} ed. New York, NY, USA: McGraw-Hill, 1964, pp. 15-64.
[Online]. Available:
\underline{http://www.bookref.com}.

\bibitem{b11} \emph{The Founders' Constitution}, Philip B. Kurland
and Ralph Lerner, eds., Chicago, IL, USA: Univ. Chicago Press, 1987.
[Online]. Available: \underline{http://press-pubs.uchicago.edu/founders/}

\bibitem{b12} The Terahertz Wave eBook. ZOmega Terahertz Corp., 2014.
[Online]. Available:
\underline{http://dl.z-thz.com/eBook/zomega\_ebook\_pdf\_1206\_sr.pdf}. Accessed on: May 19, 2014.

\bibitem{b13} Philip B. Kurland and Ralph Lerner, eds., \emph{The
Founders' Constitution.} Chicago, IL, USA: Univ. of Chicago Press,
1987, Accessed on: Feb. 28, 2010, [Online] Available:
\underline{http://press-pubs.uchicago.edu/founders/}

\bibitem{b14} J. S. Turner, ``New directions in communications,'' \emph{IEEE J. Sel. Areas Commun}., vol. 13, no. 1, pp. 11-23, Jan. 1995.

\bibitem{b15} W. P. Risk, G. S. Kino, and H. J. Shaw, ``Fiber-optic frequency shifter using a surface acoustic wave incident at an oblique angle,'' \emph{Opt. Lett.}, vol. 11, no. 2, pp. 115--117, Feb. 1986.

\bibitem{b16} P. Kopyt \emph{et al., ``}Electric properties of graphene-based conductive layers from DC up to terahertz range,'' \emph{IEEE THz Sci. Technol.,} to be published. DOI: 10.1109/TTHZ.2016.2544142.

\bibitem{b17} PROCESS Corporation, Boston, MA, USA. Intranets:
Internet technologies deployed behind the firewall for corporate
productivity. Presented at INET96 Annual Meeting. [Online].
Available: \underline{http://home.process.com/Intranets/wp2.htp}

\bibitem{b18} R. J. Hijmans and J. van Etten, ``Raster: Geographic analysis and modeling with raster data,'' R Package Version 2.0-12, Jan. 12, 2012. [Online]. Available: \underline {http://CRAN.R-project.org/package=raster} 

\bibitem{b19} Teralyzer. Lytera UG, Kirchhain, Germany [Online].
Available:
\underline{http://www.lytera.de/Terahertz\_THz\_Spectroscopy.php?id=home}, Accessed on: Jun. 5, 2014

\bibitem{b20} U.S. House. 102\textsuperscript{nd} Congress, 1\textsuperscript{st} Session. (1991, Jan. 11). \emph{H. Con. Res. 1, Sense of the Congress on Approval of}  \emph{Military Action}. [Online]. Available: LEXIS Library: GENFED File: BILLS

\bibitem{b21} Musical toothbrush with mirror, by L.M.R. Brooks. (1992, May 19). Patent D 326 189 [Online]. Available: NEXIS Library: LEXPAT File: DES

\bibitem{b22} D. B. Payne and J. R. Stern, ``Wavelength-switched pas- sively coupled single-mode optical network,'' in \emph{Proc. IOOC-ECOC,} Boston, MA, USA, 1985, pp. 585--590.

\bibitem{b23} D. Ebehard and E. Voges, ``Digital single sideband detection for interferometric sensors,'' presented at the \emph{2\textsuperscript{nd} Int. Conf. Optical Fiber Sensors,} Stuttgart, Germany, Jan. 2-5, 1984.

\bibitem{b24} G. Brandli and M. Dick, ``Alternating current fed power supply,'' U.S. Patent 4 084 217, Nov. 4, 1978.

\bibitem{b25} J. O. Williams, ``Narrow-band analyzer,'' Ph.D. dissertation, Dept. Elect. Eng., Harvard Univ., Cambridge, MA, USA, 1993.

\bibitem{b26} N. Kawasaki, ``Parametric study of thermal and chemical nonequilibrium nozzle flow,'' M.S. thesis, Dept. Electron. Eng., Osaka Univ., Osaka, Japan, 1993.

\bibitem{b27} A. Harrison, private communication, May 1995.

\bibitem{b28} B. Smith, ``An approach to graphs of linear forms,'' unpublished.

\bibitem{b29} A. Brahms, ``Representation error for real numbers in binary computer arithmetic,'' IEEE Computer Group Repository, Paper R-67-85.

\bibitem{b30} IEEE Criteria for Class IE Electric Systems, IEEE Standard 308, 1969.

\bibitem{b31} Letter Symbols for Quantities, ANSI Standard Y10.5-1968.

\bibitem{b32} R. Fardel, M. Nagel, F. Nuesch, T. Lippert, and A. Wokaun, ``Fabrication of organic light emitting diode pixels by laser-assisted forward transfer,'' \emph{Appl. Phys. Lett.}, vol. 91, no. 6, Aug. 2007, Art. no. 061103.~

\bibitem{b33} J. Zhang and N. Tansu, ``Optical gain and laser characteristics of InGaN quantum wells on ternary InGaN substrates,'' \emph{IEEE Photon. J.}, vol. 5, no. 2, Apr. 2013, Art. no. 2600111

\bibitem{b34} S. Azodolmolky~\emph{et al.}, Experimental demonstration of an impairment aware network planning and operation tool for transparent/translucent optical networks,''~\emph{J. Lightw. Technol.}, vol. 29, no. 4, pp. 439--448, Sep. 2011.

\end{thebibliography}

\begin{IEEEbiography}[{\includegraphics[width=1in,height=1.25in,clip,keepaspectratio]{a1.png}}]{First A. Author} (M'76--SM'81--F'87) and all authors may include 
biographies. Biographies are often not included in conference-related
papers. This author became a Member (M) of IEEE in 1976, a Senior
Member (SM) in 1981, and a Fellow (F) in 1987. The first paragraph may
contain a place and/or date of birth (list place, then date). Next,
the author's educational background is listed. The degrees should be
listed with type of degree in what field, which institution, city,
state, and country, and year the degree was earned. The author's major
field of study should be lower-cased. 

The second paragraph uses the pronoun of the person (he or she) and not the 
author's last name. It lists military and work experience, including summer 
and fellowship jobs. Job titles are capitalized. The current job must have a 
location; previous positions may be listed 
without one. Information concerning previous publications may be included. 
Try not to list more than three books or published articles. The format for 
listing publishers of a book within the biography is: title of book 
(publisher name, year) similar to a reference. Current and previous research 
interests end the paragraph. The third paragraph begins with the author's 
title and last name (e.g., Dr.\ Smith, Prof.\ Jones, Mr.\ Kajor, Ms.\ Hunter). 
List any memberships in professional societies other than the IEEE. Finally, 
list any awards and work for IEEE committees and publications. If a 
photograph is provided, it should be of good quality, and 
professional-looking. Following are two examples of an author's biography.
\end{IEEEbiography}

\begin{IEEEbiography}[{\includegraphics[width=1in,height=1.25in,clip,keepaspectratio]{a2.png}}]{Second B. Author} was born in Greenwich Village, New York, NY, USA in 
1977. He received the B.S. and M.S. degrees in aerospace engineering from 
the University of Virginia, Charlottesville, in 2001 and the Ph.D. degree in 
mechanical engineering from Drexel University, Philadelphia, PA, in 2008.

From 2001 to 2004, he was a Research Assistant with the Princeton Plasma 
Physics Laboratory. Since 2009, he has been an Assistant Professor with the 
Mechanical Engineering Department, Texas A{\&}M University, College Station. 
He is the author of three books, more than 150 articles, and more than 70 
inventions. His research interests include high-pressure and high-density 
nonthermal plasma discharge processes and applications, microscale plasma 
discharges, discharges in liquids, spectroscopic diagnostics, plasma 
propulsion, and innovation plasma applications. He is an Associate Editor of 
the journal \emph{Earth, Moon, Planets}, and holds two patents. 

Dr. Author was a recipient of the International Association of Geomagnetism 
and Aeronomy Young Scientist Award for Excellence in 2008, and the IEEE 
Electromagnetic Compatibility Society Best Symposium Paper Award in 2011. 
\end{IEEEbiography}

\begin{IEEEbiography}[{\includegraphics[width=1in,height=1.25in,clip,keepaspectratio]{a3.png}}]{Third C. Author, Jr.} (M'87) received the B.S. degree in mechanical 
engineering from National Chung Cheng University, Chiayi, Taiwan, in 2004 
and the M.S. degree in mechanical engineering from National Tsing Hua 
University, Hsinchu, Taiwan, in 2006. He is currently pursuing the Ph.D. 
degree in mechanical engineering at Texas A{\&}M University, College 
Station, TX, USA.

From 2008 to 2009, he was a Research Assistant with the Institute of 
Physics, Academia Sinica, Tapei, Taiwan. His research interest includes the 
development of surface processing and biological/medical treatment 
techniques using nonthermal atmospheric pressure plasmas, fundamental study 
of plasma sources, and fabrication of micro- or nanostructured surfaces. 

Mr. Author's awards and honors include the Frew Fellowship (Australian 
Academy of Science), the I. I. Rabi Prize (APS), the European Frequency and 
Time Forum Award, the Carl Zeiss Research Award, the William F. Meggers 
Award and the Adolph Lomb Medal (OSA).
\end{IEEEbiography}

\EOD

\end{document}